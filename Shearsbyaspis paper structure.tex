\documentclass[12pt,letterpaper]{article}

\usepackage{caption} % for the figure captions
\usepackage[osf]{mathpazo} % a nicer font
\usepackage{amsmath} % package for equations
\usepackage{url} % package for urls
\usepackage{hyperref} % for hyperlinks
\usepackage{graphicx} % for the figures
\usepackage{apalike}
\let\bibhang\relax
\usepackage{natbib}
\bibliographystyle{apalike}
\pagenumbering{arabic} % stating the page number type
\usepackage[usenames,dvipsnames]{color}

\usepackage{titlesec}

\titleformat*{\section}{\large\bfseries}
\titleformat*{\subsection}{\normalsize\bfseries}
\titleformat*{\subsubsection}{\normalsize}
\titleformat*{\paragraph}{\normalsize\slshape}


\usepackage{geometry} 
 \geometry{
 a4paper,
 total={210mm,297mm},
 left=20mm,
 right=20mm,
 top=10mm,
 bottom=20mm,
 }
 \linespread{1.6}

\begin{document}

\bigskip 

\begin{center} 

\noindent{\Large{\textbf{The braincase and brain cavity of the Petalichthyid \textit{Shearbyaspis oepiki} Young 1985 (Placodermi, Early Devonian)}}}
\bigskip

\noindent{Marco Castiello $^1$$^*$ and Martin Brazeau$^1$}\\

\noindent{\small{\textit{
$^1$Department of Life Sciences, Imperial College London, Silwood Campus, United Kingdom\\ % the double backslash (\\) is a short cut for 
}}} 
\end{center}
\noindent * \textbf{Corresponding author.} \textit{\small{Department of Life Sciences, Imperial College London, Silwood Campus, Buckhurst Road, Ascot SL5 7PY, United Kingdom; E-mail: m.castiello13@ic.ac.uk}}


\newpage
\begin{abstract}
Fossil stem-gnathostomes offer the potential to fill the morphological gap between extant jawless and jawed vertebrates, allowing us to reconstruct the stepwise evolution of the gnathostome body plan. The petalichthyids are placoderm-grade stem gnathostomes thought to possess jaws, but also display a combination of neurocranial features otherwise found only in jawless vertebrates, particularly osteostracans. Because of this, they have become central to the debate on the relationships of placoderms and the primitive cranial architecture of gnathostomes. Nevertheless, only the braincase of the petalichthyid \textit{Macropetalichthys} has been studied in detail and the diversity of neurocranial morphology in this group remains poorly documented. Using xray computed microtomography, we investigate the endocranial morphology of \textit{Shearsbyaspis oepiki}, Young 1980, a petalichthyid from the Early Devonian of Taemas-Wee Jasper, Australia. We generated virtual three-dimensional reconstructions of the external endocranial surfaces, orbital walls, and cranial endocavity, including canals for major nerves and blood vessels. The neurocranium of \textit{Shearsbyaspis} resembles \textit{Macropetalichthys} in particular in the course of nerves and blood vessels. Many characters, as the morphology of the pituitary vein canal and the course of the trigeminal nerve, recall the morphology of osteostracans and mark a substantial difference with the rest of the placoderms. These contrast with the discovery of a parasphenoid, a character that might indicate petalichthyids as close relatives of arthrodires, raising some questions about current proposals of placoderm paraphyly. Our detailed description of these specimens provides more information about the cranial and brain anatomy of petalichthyids, allowing to reinvestigate the phylogenetic significance of petalichthyid cranial morphology and comparing the strengths and weaknesses of competing scenarios of placoderm relationships.
\end{abstract}

\noindent \textbf{Key words:} stem-gnathostomes, early vertebrates, neurocranium, endocast, homology, osteostracans

\newpage 

\section{Introduction}

\textcolor{MidnightBlue}{Can I start straight to the point introducing shearsbyaspis? I mean, for the thesis chapter, sure...but for a paper?}

\textcolor{Red}{OUTLINE}

\textit{Shearsbyaspis oepiki} is a petalichthyid placoderms from Australia.... It is known from these specimens.

They have been founds there...

It has been studied before by

Why it is important?

What we did with it and what we are going to talk about in the rest of the paper.

\section{Material and Methods}

The following description of the neurocranium of \textit{Shearsbyaspis oepiki} \citealt{Young1985} is based on holotype P33580, held in the Natural History Museum of London (NHM), an incomplete skull-roof with endocranial ossification. It has been collected in the Murrumbidgee Group of the Taemas Formation, in New South Wales, Australia, dated to the Early Devonian (Emsian), around 405-395 million years ago \citep{Young1978,Young1980,Young1985}. It  was originally described and named by Young in 1985 after chemical preparation with acetic acid, following \cite{Toombs1948} and \cite{Toombs1959}. 

\textcolor{Red}{NOTE: Can the paragraph above go in the introduction and not being repeated here? I mean, here we just remind which specimen are we using}.

Modern tomographic technique (CT scan) was used to investigate the internal anatomy of this well preserved tridimensional specimen, which was impossible to ascertain when it was firstly described. It was scanned at the Imaging and Analysis Center at the Natural History Museum in London, using X-ray micro-CT scanner. Scanning parameters are: 180 kV; 180 µA; 0.1 mm thick copper filter; 6284 projections and Voxel size 15.3 µm. The resulting images were rendered and segmented using the software Mimics 15.01 (Materialise Technologielaan, Leuven, Belgium).



\section{Description}
\subsection {Skull roof}
\subsubsection {Preservation and general features}

P33580 consists of an incomplete skull with most of the underlying neurocranial ossification preserved. The anterior most and posterior most area are damaged so that the tip of the rostral plate and the area behind the otic region are missing. The dorsal side of the skull roof is exposed and well preserved, with clear pores of the lateral line system, the pineal foramen and the orbits. The visceral surface of the skull-roof is visible from the CT data, display the promined enclosed tubes containing the sensory canals. 

The skull-roof has a slightly convex axial profile. In dorsal view it appears convex, with a shallow embayment on the lateral margin just anterior to the orbits.The orbits are dorsolaterally placed and display an anterodorsal torus, so that between the orbits the skull is slightly concave. The different plates of the skull-roof are not really clear but their margins can be inferred from some sutures and the pattern of the ornamentations and ossification centres. The skull-roof pattern (Fig. \ref{sensorydrawing}) has already been described and figured by \cite{Young1985} (pag 124, figure 4; pag 125, figure 5). 

The dermal ornamentation consists of low concentric or radiating ridge, with rounded tubercles irregularly located along them. This type of ornamentation is unique of \textit{Shearsbyaspis} and represents a diagnostic feature of this taxon \citep{Young1985}.

\begin{figure}[!h]
\centering
    \includegraphics[scale=0.6]{Figures/ShearsbyaspisSensory.jpg}
\caption{\footnotesize{Restoration of the dermal bone and sensory lines of \textit{Shearsbyaspis oepiki}. Redrawn from \citealt{Young1985}}}
\label{sensorydrawing}
\end{figure}

\subsubsection{Sensory line canals}

The sensory lines canals of P33580 are well developed and clearly distinguishable on the dermal armour and illustrated in figure \ref{sensory3D}. They consist of enclosed tubes situated deeply into the bones, opening to the exterior through a single row of large pores, as common in the other petalichthyids and in ptyctodontids ( e.g. \citealt{Stensio1925,Miles1967,Denison1978,Young1978,Young1985,Forey1986a,Zhu1991,Trinajstic2012,pan2015new}). Due to the incompleteness of the specimen, only the supraorbital sensory canals (soc) and the infraorbital sensory line (ioc) are preserved.

The supraorbital sensory canals (soc) are clearly defined, opening with pores situated in the external dorsolateral surface of the sensory tubes. The canals show a different morphology depending on their position on the dermal plates. For instance, in the rostral part of the skull, anterior to the eyes, each canal open in a broad groove deeply sunk into the bone, while near and poterior to the eyes they opens as a row of distinct rounded pores shallower sunk in the dermal surface.   A similar condition is present in \textit{Notopetalichthys} \cite{Woodward1941} and \textit{Ellopetalichthys} \cite{Ørvig1957}. Interestingly, the pores merged in a small grove at the contact with the nuchal plates (gr.nu), like in the ptyctodontids \textit{Materpiscis} \cite{Trinajstic2012}, before continuing posteriorly as pores.

The infraorbital canals (ioc) extend along the lateral margin of the postorbital plates and open to the exterior as a single row of pores. The pores are more regular in the posterior part of this sensory line, becoming a single narrow grove near the anterior part of the orbit. All the pores opening are laterally placed but one opens more dorsally, on the posterior part of the external orbital wall. As revealed by the CT scan, this pore is related to of an area where numerous short and small canals connect the orbit to the infraorbital canal  \textcolor{MidnightBlue}(still really need to think about what that is)(there might a functional meaning for that, but I still need to think about it.)

\begin{figure}[!h]
\centering
    \includegraphics[scale=0.8]{Figures/shearsbyaspis3d.png}
\caption{\footnotesize{3D Restoration of the dermal bone and sensory lines of P33580.}}
\label{sensory3D}
\end{figure}

A more complete evaluation of the sensory line system of \textit{Shearsbyaspis} can be made comparing both the two available specimens, P33580 and  CPC24622, as shown in Young (\citealt{Young1985} p. 125, fig. 5). According to his reconstruction (Fig. \ref{sensorydrawing}), both the supraorbital sensory canals (soc) and the posterior pitline (ppl) converge together on the ossification centre of the nuchal plates, with anastomotic posterior pitline canals not in contact with the supraorbitals sensory canals.

The infraorbital lateral lines (ioc) passes over the postorbital plate, merging the postmarginal canal (pmc) and the main lateral line (mc) on the marginal plate. The latter is fairly long and run posteriorly on the paranuchal, merging with the posterior pit lines on the centre of ossification of the anterior paranuchal, as in other petalichthyids. 

\subsection{Endocranium}
\subsubsection{General features}

Previous acid preparation of the specimen by \citep{Young1985} has partially exposed the ventral and lateral external neurocranial surfaces, the left orbit and part of the perichondral tissue surrounding some of the internal cavities and canals, so that Young was able to observe some details of the braincase morphology of P33580. any details remains obliterated by the surrounding rock matrix and were not observable at that time, but thanks to this preservation and the previous acid preparation, they have been finally resolved in our CT scan \ref{ventral3D}. Integrating both external observations and CT data it is possible to have a detailed description of the endocranial anatomy of P33580.

\begin{figure}[!h]
\centering
    \includegraphics[scale=0.05]{Figures/shearsbyaspisbetterlightstif.png}
\caption{\footnotesize{3D restoration of the dermal bone (light orange), endocranium (white) and endocast (viridian) of P33580, ventral view.}}
\label{ventral3D}
\end{figure}

The neurocranium can be divided in different layers, as recognized by \cite{Stensio1925} for \textit{Macropetalichthys}. A thin perichondral layer surrounds the cerebral cavity, separating it from the rest of the braincase in what \cite{Stensio1925} called cavum cerebrale. It extends between the orbits and an interorbital septum is not present, the braincase could be defined as platybasic \citep{maisey2007braincase}. The braincase is bordered by a thin layer of perichondral bone on the ventral and lateral side, while on the dorsal side a perichondral layer seems to be fused with the skull-roof or very thin, so that the separation with the overlaying dermal bone cannot be detected. Perichondral tissue is also present around all the intramural canals, permitting to follow the course of nerves and vessel. No evidence of intracranial divisions could be detected, and the endocranium is preserved as a single ossified structure.

In overall shape the neurocranium is broad and flat, with a deep concave transverse profile, generally enlarging anteroposteriorly being elongated and narrower at the front and wider in the optic region. The ethmoid region is not well preserved so that the position of the nasal capsule cannot be detected. The dorsal external surface of the neurocranium cannot be seen, as the limits between it and the visceral surface of the skull-roof cannot be clearly distinguished. On the ventral side \ref{ventralreconstr}, the external endocranial surface is transversely concave over its entire length. On its surface, several groves and foramina could be recognized, signalling the passage of blood vessels and branches of nerves. A parasphenoid is visible, showing the foramen for the hypophysial duct, and is associated with paired hypophysial depressions on the ventral endocranial surface.

\begin{figure}[!h]
\centering
    \includegraphics[scale=0.25]{Figures/BraincaseVentralReconstr.png}
\caption{\footnotesize{Drawing of the dermal bone (white) and endorcranium (gray) of P33580 as preserved (left) and restored (right)}}
\label{ventralreconstr}
\end{figure}

\subsubsection{Orbital area}

The orbits are large and surrounded ventrolaterally by a well-developed subocular shelf and present anteriorly a large preorbital space, as observed in other petalichthyid-like taxa as \textit{Macropetalichthys} \citep{Stensio1925,Stensi1963b,Stensio1969} and \textit{Brindabellaspis} \citep{Young1980}. Differently from them, the preorbital space is completely separated from the orbital space by a well-developed infraorbital ridge so that any preorbital fenestra is detected. 

The mesial wall of the orbit present a large oval opening, surrounded by an everted rim of pericondral bone, comunicating with the interpericontral space and representing the area for the attachment of the eye stalk. Anterodorsally the eyes stalk attachment, a big fenestra is related the opening of the optic nerve (N.II). Near the dorsal margin of this opening, a shallow recess extends posteromedially on the mesial wall (my3). It is pierced by small foramina, possibily representing branches of the oculomotor nerve (N. III) so that this recess could correspond with the myodomes for the muscles connected with this nerve. Another small foramen it is visible on the anterior corner of the orbital wall, connected with a canal \textcolor{MidnightBlue}{(still really need to think about what that is)}. 

The posterodorsal part of the the mesial orbital wall is pierced by several small foramina surrounding a bigger foramen. They represent the opening for the ramus opthalmicus lateralis of the facial nerve. Anterodorsally, a small foramina represent the opening for the throclear nerve. A big foramen opens in the posteroventral corner of the orbital wall and became a sack-like recess. The recess is pierced by three different foramen in its posteroventral end, representing the opening for the pituitary vein and the branches of the trigeminal nerve (N. V). The posterodrosal end of the reccess is blind so that it suggestive to have contained a myodome \textcolor{MidnightBlue}{(maybe)}. Dorsally to this big foramen, another opening on the orbital wall represents the foramen for the abduscens nerve (N.IV). The posterodorsal corner of the orbital wall is pierced by several small foramina possibily for blood vessels.

The subocular shelf is well developed and has a distinct groove and foramen on its ventral surface for the course of the efferent pseudobranchial artery. Medially, several foramina signal the passage of canals, probably representing blood vessels. One foramen is associated with the palatine branch of the facial nerve (N.VII).

\begin{figure}[!h]
\centerline{\includegraphics[scale=0.115]{Figures/shearsbyaspisLatDraw.png}}
\caption{\footnotesize{Drawing of right lateral surface of the braincase of P33580}}
\label{lateral}
\end{figure}

\subsubsection{Postorbital region}

The postorbital region of P33580 is partially preserved, with only the right portion of the region covering the labyrith preserved. This limits the amount of data that can be infered of the postorbital and in general the posteriormost region of the braincase of \textit{Shearasbyapis}. A tentative reconstruction has been made mirroring the right portion of the braincase. The postorbital region of the neurocranium appear like a wide unpierced surface of perichondral bone, covering the brain cavity, the labyrith and associated canals. No information of the occipital area can be provided.

\subsection{Endocranial cavity}

\subsubsection{Cerebral cavity}

The good preservation of the cerebral cavity allows reproducing the endocast and investigating the brain cavity morphology. The endocast is not complete, missing the occipital region and part of the area near the labyrinth. Nevertheless, the different region of the brain, the forebrain, the midbrain and the hindbrain, could be detected using landmarks as the canals for the nerves and the position of foramina and glands. In general the brain cavity appears long and narrow, with anteriorly directed elongated olfactory ducts, in a way similar to that of \textit{Macropetalichthys} (Stensiӧ, 1925, 1963, 1969). 

\paragraph{Forebrain}

The forebrain is constituted by the telencephalon and the diencephalon. The limit between these two regions is marked by the anterior margin of the foramen for the optic nerve (II), originating from the diencephalon. The telencephalon represents the first division, carrying the terminal (N. 0) and olfactory nerves (N. I). In \textit{Shearsbyaspis}, as in \textit{Macropetalichthys}, it is subdivided in two well-developed olfactory lobes, clearly distinct both dorsally and ventrally. The ventral portion of the olfactory lobes seems to be longer, extending posteriorly until the hypophysial canal. 

The diencephalon is stout and very wide, carrying the optic nerve (N. II), the pineal gland and the hypophysis. Its limit with the subsequent mesencephalic region is not very clear in dorsal view; while on the ventral part could be places behind the hypophysial duct. In ventral view, the diencephalon of Shearsbyaspis seems different from that of Macropetalichthys in lacking two well differentiated lobes in the hypothalamus and in the shape of the pineal organs, being lozenge-shape and asymmetrical instead of tubular and wider at the base. Two ridges associated with the central body of the pineal gland could be related with the parietal organ. The pineal organ opens on the dermal skull through a foramen between the anterior parts of the orbits. The hypophysial duct is placed slightly posteriorly in respect of the pineal gland and project anteroventrally, opening through the palate with a single foramen in the parasphenoid. 

\paragraph{Midbrain}

The anterior limit of the midbrain is not clear, but could be placed behind the optic nerve and the hypophysial duct. Its posterior edge, where it meets the cerebellar auricles of the metencephalon, is marked by an evident constriction. In P33580 the dorsal part of the brain cavity is broken, so that is not possible to have a precise picture of that area. Anyhow, it seem that distinct optic lobes where not present. Interestingly, the midbrain carries only the oculomotor nerve (III), while the trochlear nerve is instead associated with the cerebellum.

\paragraph{Hindbrain}

The hindbrain is well preserved so that the bounds of his two divisions, the metencephalon and the myelencephalon, could be easily detected. The metencephalon is constituted by two symmetrical rounded lobes, the auricular recesses of the cerebellum, and a straight myelencephalic portion. The limit between these two divisions is marked by the posterior edge of the metencephalic lobes, where the brain become narrower and the facial nerve (N. VII) originated. From the metencephalon originate the trochlear nerve (N.IV) , the trigeminal (N. V) and abducens nerves (N. VI). The myelencephalon is the point of origin of the remaining cranial nerves and runs posteriorly until the end of the brain cavity. Unfortunately, in this specimen is only preserved the incomplete left  anterior part.

\paragraph{Labyrinth cavity}

Parts of both the labyrinth cavities are preserved in P33580, even if none of them is complete. The right labyrinth shows the position and morphology of the vestibule apparatus and related cranial nerves entering in it. As in \textit{Macropetalichthys}, the saccular portion is very big compare to other placoderms, being as big as the orbits, if not more.  Furthermore, the orbit and the labyrinth are very nearly placed, almost in contact each other, as in osteostracans, so that the facial nerve passes so close to the labyrinth, before entering in the ventral posterior orbit, that there is almost indistinguable from the saccular wall. 
The anterior ampulla is placed oddly ahead, behind the posterior external side of the orbit, so that the anterior semicircular circular is very long. The latter crosses the sacculus running towards the centre and enters in it above the point of connection between the labyrinth and the acoustic nerve (N. VIII). The anterior ampulla is connected to the external ampulla by a ventrally bended utriculus. Unfortunately the posteriormost part of the labyrinth is not preserved so it is not possible to observe the external semicircular canal nor the posterior ampulla and canal, as well the endolymphatic duct.

\subsubsection{Cranial nerves}
\paragraph{Terminal, olfactory and optic nerve}

\paragraph{Oculomotor, trochlear and abduscens nerves}
\paragraph{Trigeminal nerve}
\paragraph{Facial nerve}
\paragraph{Acoustic and glossopharyngeal nerves}

\subsubsection{Cranial vessels}
\paragraph{Arteries}
\paragraph{Veins}

\section{Discussion}
(titles are still provisional)
\subsection{Dermal bone pattern Petalichthyida and in early gnathostomes}

\subsection{Cranial nerve patter and gnathostomes primitive conditions}


\subsection{Similarities between osteostracans and petalichthyida and the monophyly of placoderms}

\section{Conclusion}

\section{Acknowledgements}

\bibliographystyle{apalike}
\bibliography{References}

\end{document}