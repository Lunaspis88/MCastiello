\documentclass[12pt,letterpaper]{article}

\usepackage{caption} % for the figure captions
\usepackage[osf]{mathpazo} % a nicer font
\usepackage{amsmath} % package for equations
\usepackage{url} % package for urls
\usepackage{hyperref} % for hyperlinks
\usepackage{graphicx} % for the figures
\usepackage{apalike}
\let\bibhang\relax
\usepackage{natbib}
\bibliographystyle{apalike}
\pagenumbering{arabic} % stating the page number type

\usepackage{geometry} 
 \geometry{
 a4paper,
 total={210mm,297mm},
 left=20mm,
 right=20mm,
 top=10mm,
 bottom=20mm,
 }
 \linespread{1.6}

\begin{document}

\bigskip 

\begin{center} 

\noindent{\Large{\textbf{The braincase and brain cavity of the Petalichthyid \textit{Shearbyaspis oepiki} Young 1985 (Placodermi, Early Devonian)}}}
\bigskip

\noindent{Marco Castiello $^1$$^*$ and Martin Brazeau$^1$}\\

\noindent{\small{\textit{
$^1$Department of Life Sciences, Imperial College London, Silwood Campus, United Kingdom\\ % the double backslash (\\) is a short cut for 
}}} 
\end{center}
\noindent * \textbf{Corresponding author.} \textit{Department of Life Sciences, Imperial College London, Silwood Campus, Buckhurst Road, Ascot SL5 7PY, United Kingdom; E-mail: m.castiello@ic.ac.uk}

\bigskip

\begin{abstract}

Fossil stem-gnathostomes offer the potential to fill the morphological gap between extant jawless and jawed vertebrates, allowing us to reconstruct the stepwise evolution of the gnathostome body plan. The petalichthyids are placoderm-grade stem gnathostomes thought to possess jaws, but also display a combination of neurocranial features otherwise found only in jawless vertebrates, particularly osteostracans. Because of this, they have become central to the debate on the relationships of placoderms and the primitive cranial architecture of gnathostomes. Nevertheless, only the braincase of the petalichthyid \textit{Macropetalichthys} has been studied in detail and the diversity of neurocranial morphology in this group remains poorly documented. Using xray computed microtomography, we investigate the endocranial morphology of \textit{Shearsbyaspis oepiki}, Young 1980, a petalichthyid from the Early Devonian of Taemas-Wee Jasper, Australia. We generated virtual three-dimensional reconstructions of the external endocranial surfaces, orbital walls, and cranial endocavity, including canals for major nerves and blood vessels. The neurocranium of \textit{Shearsbyaspis} resembles \textit{Macropetalichthys} in particular in the course of nerves and blood vessels. Many characters, as the morphology of the pituitary vein canal and the course of the trigeminal nerve, recall the morphology of osteostracans and mark a substantial difference with the rest of the placoderms. These contrast with the discovery of a parasphenoid, a character that might indicate petalichthyids as close relatives of arthrodires, raising some questions about current proposals of placoderm paraphyly. Our detailed description of these specimens provides more information about the cranial and brain anatomy of petalichthyids, allowing to reinvestigate the phylogenetic significance of petalichthyid cranial morphology and comparing the strengths and weaknesses of competing scenarios of placoderm relationships.

\end{abstract}

\noindent \textbf{Key words:} stem-gnathostomes, early vertebrates, neurocranium, endocast, homology, osteostracans

\bigskip

\newpage 
\section{Introduction}
Displaying a combination of neurocranial trait recalling both jawed and jawless vertebrates, and remarkable unique features, as their spoon-like dermal neck joint, the Petalichthyida occupy a special position among placoderms \citep{Janvier1996a,brazeau2009braincase,brazeau2014characters,Janvier2015,brazeau2015origin}. Nevertheless, relatively only with few petalichthyids have been included in the most recently published analyses on stem-gnathostomes \citep{brazeau2009braincase,davis2012,zhu2013silurian,long2015copulation,giles2015osteichthyan}, with most of placoderms diversity represented by arthrodires. This is mainly because to date only very few petalichthyid neurocranial remains have been published \citep{Stensi1925,Stensi1969,Young1978,pan2015new}, limiting the possibility of including petalichtyids characters and taxa diversity in phylogenetic analysis. As a result, their contribution to the study of the stepwise evolution of modern vertebrate remain unclear.

Being present in every vertebrate, the study of the braincase and its internal structure have remarkable potential in vertebrate systematics.This is particularly important when the study of the endocranium makes possible to draw comparison across groups with substantially different external anatomy, as in the case of the stem-gnathostomes. Recent increase in the use of non invasive tomographic techniques has provided a crucial tools the study the internal anatomy of fossil specimens, filling many gaps in our knowledge about endocranial diversity in the different group of steam-gnathostome. 

Applying modern computed tomographic techniques, here we described the neurocranial anatomy of the petalichthyids \textit{Shearsbyaspis oepiki}. As the first and only description of a petalichthys braincase was done with invasive mechanical tecniques, as serial grinding, this will be a good chance to compare previous interpretation made by stensio and to refine and improve our knowledge about petalichthyida. the incorporation of a new petalichthyid and characters pertaining to endocast morphology of these animals in future cladistics analysis (to be continued).


\section{Material and Methods}
The following description of the neurocranium of \textit{Shearsbyaspis oepiki} \citealt{Young1985} is based on holotype P33580, held in the Natural History Museum of London (NHM), an incomplete skull-roof with endocranial ossification. It has been collected in the Murrumbidgee Group of the Taemas Formation, in New South Wales, Australia, dated to the Early Devonian (Emsian), around 405-395 million years ago (CIT NEEDED). It  was originally described and named by Young in 1985 after chemical preparation with acetic acid, following \cite{Toombs1948} and \cite{Toombs1959}. 
Modern tomographic technique (CT scan) was used to investigate the internal anatomy of this well preserved tridimensional specimen, which was impossible to ascertain when it was firstly described. It was scanned at the Imaging and Analysis Center at the Natural History Museum in London, using X-ray micro-CT scanner. Scanning parameters are: 180 kV; 180 µA; 0.1 mm thick copper filter; 6284 projections and Voxel size 15.3 µm. The resulting images were rendered and segmented using the software Mimics 15.01 (Materialise Technologielaan, Leuven, Belgium).

\section{Description}
\subsection {Skull roof}
\subsubsection {Preservation and general features}

P33580 consists of an incomplete skull with most of the underlying neurocranial ossification preserved. The anterior most and posterior most area are damaged so that the tip of the rostral plate and the area behind the otic region are missing. The dorsal side of the skull roof is exposed and well preserved, with clear pores of the lateral line system, the pineal foramen and the orbits. 
The skull-roof has a slightly convex axial profile. In dorsal view it appears convex, with a shallow embayment on the lateral margin just anterior to the orbits. The orbits are dorsolaterally placed and display an anterodorsal torus, so that between the orbits the skull is slightly concave. This morphology is shared by other petalichthyids like \textit{Macropetalichthys} \citep{Stensi1925}, \textit{Notopetalichthys} \citep{Woodward1941} and \textit{Holopetalichthys} \citep{Zhu1996a}. 

The dermal ornamentation consists of low concentric or radiating ridge, with rounded tubercles irregularly located along them. This type of ornamentation is unique of \textit{Shearsbyaspis} and represents a diagnostic feature of this taxon \citep{Young1985}. The visceral surface of the skull-roof cannot be seen, as it is obscure by the dorsal surface of the braincase.

The different plates of the skull-roof are fused together so that their margins can only be inferred from the sutures and the pattern of the ornamentations and ossification centres. The skull-roof pattern (Fig. 1) has already been described and figured by \cite{Young1985} (pag 124, figure 4; pag 125, figure 5).

\subsubsection{Sensory line canals}

The sensory lines canals are well developed and clearly distinguishable on the dermal armour. They consist in tubes situated deeply into the bones, opening to the external through a single row of large pores, as common in the other petalichthyids and in ptyctodontids ( e.g. \citealt{Stensi1925,Miles1967,Denison1978,Young1978,Young1985,Forey1986a,Zhu1991,Trinajstic2012,pan2015new}). Unfortunately, due to the incompleteness of the examined specimen, only the supraorbital sensory canals (soc) and the infraorbital sensory line (ioc) are preserved (Fig X).

The supraorbital sensory canals are pronounced, with pores opening on the lateral side of the canal. They show a different morphology depending on their position on the dermal plates. For instance, near and posterior to the eyes the pores are rounded, placed close together and at a fairly regular distance. As a contrast, in the rostral part of the head, anterior to the eyes, they are merged together and indiscernible so that the canal runs into a laterally open grove (gr.soc). A similar condition is present in \textit{Notopetalichthys} \cite{Woodward1941} and \textit{Ellopetalichthys} \cite{Ørvig1957}. Interestingly, the pores merged in a small grove at the contact with the nuchal plates (gr.nu), like in the ptyctodontids \textit{Materpiscis} \cite{Trinajstic2012}, but they turn back to be single rounded opening behind this point. The supraorbital canals is innervated by various dorsal branches of the ramus ophthalmicus lateralis, raising from the facial nerve and running along the internal lateral wall of the orbits, as in \textit{Macropetalichthys} \cite{Stensi1925,Stensi1963b,Stensi1969}.

The infraorbital canals (ioc) run along the lateral margin of the postorbital plates and open to the exterior with a single row of pores. The pores are more regular in the posterior part of this sensory line, becoming a single narrow grove near the anterior part of the orbit. All the pores opening are laterally placed but one opens more dorsally, on the posterior part of the external orbital wall. As revealed by the CT scan, this pore marks the starting point of an area where numerous short and small canals connect the orbit to the infraorbital canal. A similar series of small canals branching out of the infraorbital canal is visible also in \textit{Macropetalichthys}, and has been suggested by \citealt{Stensi1925} to be transmitted by the ramus oticus lateralis of the facial nerve. Our CT scan hasn’t revealed any nerve passing in this position, but, instead, these branches run directly and independently from the infraorbital canals to the orbits. Unfortunately, it has been impossible to review these details in the specimens analysed by Stensi{\o}.

A more complete evaluation of the sensory line system of \textit{Shearsbyaspis} can be made comparing both the two available specimens, P33580 and  CPC24622, as shown in \citep{Young1985}. According to his reconstruction (\citealt{Young1985}, p. 125, fig. 5), both the supraorbital sensory canals (soc) and the posterior pitline (ppl) converge together on the ossification centre of the nuchal plates (figure X). This features is also present in other petalichthyids, in most of the ptyctodontids, as well as some acanthothoracids and arthrodires (e.g. \citealt{Denison1978,Young1980,Dupret2009,dupret2011skull,Trinajstic2012}). The pattern of this convergence is very disparate in petalichthyids, as discussed later. In \textit{Shearsbyaspis}, the arrangement is similar to that of \textit{Notopetalichthys} \citep{Woodward1941}, \textit{Wijdeaspis} \citep{Young1978} and \textit{Epipetalichthys} \citep{Stensi1925}, with anastomotic posterior pitline canals not in contact with the supraorbitals sensory canals \citep{Young1985}.

Regarding the infraorbital lateral lines (ioc), it passes over the postorbital, merging the postmarginal canal (pmc) and the main lateral line (mc) on the marginal plate. The latter is fairly long and run posteriorly on the paranuchal, merging with the posterior pit lines on the centre of ossification of the anterior paranuchal, as in other petalichthyids. On this plate, an opening visible on the visceral surface of the specimen CPC24622 has been associated with the opening for the endolymphatic duct (Young 1985, p. 124, fig 4.). The central sensory line is absent, a feature regarded as diagnostic for Petalichthyida (Zhu, 1991). Furthermore, there is also no sign of an occipital cross commissure, a feature considered derived for the group (Zhu, 1991; Zhu and Wang, 1996) and instead present in some arthrodire-like petalichthyids like \textit{Eurycaraspis} and \textit{Diandongpetalichthys}.

\subsection{Endocranium}
\subsubsection{General features}
Although the perichondral layer forming the endocranium is very thin and has been distorted or crushed in some areas, it is possible to observe the general aspect of the ventral and lateral sides of the braincase and related features. Some structures are not uniformly preserved but the good grade of preservation allowed reconstructing the overall morphology mirroring the most well preserved areas (Fig. X). 
In overall shape the neurocranium is broad and deep, generally enlarging anteroposteriorly being elongated and narrower at the front and wider in the optic region. The orbits are large and surrounded lateroventrally by a well-developed subocular shelf, as observed other petalichthyid-like taxa as \textit{Macropetalichthys} (Stensiӧ, 1925, 1963, 1969) and \textit{Brindabellaspis} (Young, 1980), resembles the condition of osteostracans (Janvier, 1985, 1996a; Brazeau and Friedman, 2014). The subocular shelf presents on its ventral surface a distinct groove for the course of the efferent pseudobranchial artery. Anterior to the orbits, a large preorbital space (cav.pro) is present, similar to that observable in \textit{Macropetalichthys} and \textit{Brindabellaspis}. Differently from them, it is completely separated from the orbital space by a well-developed ridge projecting anterodorsally from the subocular shelf so that any preorbital fenestra is detected. 

On the ventral side, the endocranium is transversely concave over its entire length, with the mesial portion slopping upwards in correspondence of the lateral preorbital space and the optic region, being instead more gently concave in the otic region. On its surface, several groves and foramina could be recognized, signalling the passage of blood vessels and branches of nerves (7 pal. eps, etc..). 
The neurocranium can be divided in different layers, as recognized by Stensiӧ for \textit{Macropetalichthys} (1925). A thin perichondral layer surrounds and protects the cerebral cavity, separating it from the rest of the braincase in what Stensiӧ called cavum cerebrale. It extends between the orbits and an interorbital septum is not present, the braincase could be defined as platybasic (Maisey, 2007). The braincase is bordered by a thin layer of perichondral bone on the ventral and lateral side, while on the dorsal side a perichondral layer seems to be fused with the skull-roof or very thin, so that the separation with the overlaying dermal bone cannot be detected. Perichondral tissue is also present around all the canals passing between the cerebral cavity and the endocranium. This permits to follow the course of nerves and vessel, providing a great deal of information. 

No evidence of separate ossifications could be detected, and the endocranium is preserved as a single ossified structure. This contrast with what has been seen in the “loose nose” placoderms (see Goujet, 2001). In these forms, the braincase is composed by two separate ossifications, a rhinocapsular and a postethmo-occipital portion. These two ossifications of the braincase have been found as actually separated in adult specimens, as in the acanthothoracid \textit{Romundina} and arthrodires, or as bony laminae inside a single ossified unit, as in the acanthothoracid \textit{Brindabellaspis}. The latter has been regarded as a secondary fusion, on the evidence that separate nasal capsule chondrifications are present in embryos of extant vertebrates and thus this could be the primitive state for jawed gnathostomes (Goujet, 2001). Nevertheless, a discrete perichondral division in adult braincase is absent in any other non-placoderm gnathostomes, calling caution in the analysis of the polarity and phylogenetic significance of this character (Brazeau and Friedman, 2014). Petalichthyida seem to possess a single well ossified braincase in the adult stage, as appear from the lack of any fissure or laminae in the neurocranium of \textit{Macropetalichthys} and \textit{Shearsbyaspis}. Nonetheless, in both these taxa the most anterior part of the ethmoid region is missing, as thus is not possible to identify the position of the nasal capsules. This can suggest that the nasal capsule were actually separate from the main braincase, but unfortunately it is difficult to tell if they were ossified, chondrified or just include in loose tissue. Another possible interpretation is that they might have been located in another place (see below).
\subsubsection{Nasal area}
\subsubsection{Orbital area}

The left orbit is exposed and a description has already been present in the original description \citep{Young1985}, although the CT reveals that some areas of it are broken or not preserved. The right orbit is instead filled by the matrix and this resulted in an excellent preservation, so that it is possible to reconstruct precisely the orbital wall from the CT data. 

\subsubsection{Suborbital and Postorbital region}

\subsection{Endocranial cavity}
\subsubsection{Cerebral cavity}
The good preservation of the cerebral cavity, enclosed in perichondral tissue, allows reproducing the endocast and investigating the brain morphology. The brain cast is not complete, missing the occipital region and part of the area near the labyrinth. Nevertheless, the different region of the brain could be detected using landmarks as the canals for the nerves and the position of foramina and glands. In general, the endocast present the typical fish-brain regionalization, being constituted by a forebrain, a midbrain and a hindbrain. However, it appears long and narrow, with anteriorly directed elongated olfactory ducts, in a way similar to that of \textit{Macropetalichthys} (Stensiӧ, 1925, 1963, 1969). These differ from most of the other known endocast of placoderms, where the brain and the olfactory tracts are shorter (Stensiӧ, 1963, 1969; Goujet, 1984; Zhu and Janvier, 1996; Elliott and Carr, 2010; Dupret et al., 2014). The other exception has been reported in pholidosteid arthrodires (Stensiӧ, 1969), and the elongation of the telencephalon, rather present in many crown gnathostomes, is regarded as an independent evolution of these two clades (Dupret et al., 2014). 

\paragraph{Forebrain}

The forebrain is constituted by the telencephalon and the diencephalon. The limit between these two regions is marked by the anterior margin of the foramen for the optic nerve (II), originating from the diencephalon. The telencephalon represents the first division, carrying the terminal (0) and olfactory nerves (I). In \textit{Shearsbyaspis}, as in \textit{Macropetalichthys}, it is subdivided in two well-developed olfactory lobes, clearly distinct both dorsally and ventrally. The ventral division seems to be longer, running back until the hypophysial canal. Distinct and well-developed olfactory lobes are not present in all placoderms and are probably related to the elongation of the olfactory tracts in petalichthyids. In arthrodires like \textit{Dicksonosteus} and \textit{Kujdanowiaspis} the olfactory bulbs are small and not very differentiated from the rest of the telencephalon, even though separated (Stensiӧ, 1963; Goujet, 1984). A different morphology is present in the acanthothoracid \textit{Brindabellaspis}, where the telencephalon end is round and is not possible to observe any divisions for olfactory lobes (Young, 1980). However, as will be discussed later, the position of the nostril and associated nerves in this taxon is not clear and make difficult to assign the possible position of the olfactory lobes and tract. Another acanthothoracid, Romundina, shows instead well differentiated olfactory lobes associated with big nasal sacs, even though it is different from petalichthyids in having short olfactory tract (Dupret et al., 2014). Separated and well-developed olfactory bulbs, in some way similar to that of petalichthyids, have been reported for the galeaspid Shuyu (Gai et al., 2011), providing evidence of the presence of paired nasal sacs in a jawless stem-gnathostome with an unpaired nasal opening. A rounded telencephalic region, with undifferentiated olfactory bulbs connected with a singular nasal opening, is instead present in the jawless osteostracans (Janvier, but I need to see more references)

The diencephalon is stout and very wide, carrying the optic nerve (II), the pineal gland and the hypophysis. Its limit with the subsequent mesencephalic region is not very clear in dorsal view; while on the ventral part could be places behind the hypophysial duct. In ventral view, the diencephalon of Shearsbyaspis seems different from that of Macropetalichthys in lacking two well differentiated lobes in the hypothalamus and in the shape of the pineal organs. This is tubular and wider at the base in Macropetalichthys, while in Shearsbyaspis is lozenge-shape and asymmetrical. As in Macropetalichthys, there are two ridges associated with the central body of the pineal gland, related by Stensiӧ to the parietal organ. The pineal organ opens on the dermal skull through a foramen between the anterior parts of the orbits. (a pineal foramen is visible on the specimens of Macropetalichthys figured by stensio 1925 – plates XIX, XX, even though has been scored as it has a closed pineal opening in recent matrix). È perchè c’è un pineal recess, quindi l’endocranium presenta un foro ma poi non si apre verso l’esterno nel dermal skull roof It is interesting to note that as in petalichthyids and Brindabellaspis, also in the jawless osteostracans and galeaspids the pineal gland is located above or slightly forward to the optic nerve (Young, 1980; Janvier, 1985, 1996b; Gai et al., 2011), while in arthrodires it’s placed more posteriorly (Stensiӧ, 1969; Goujet, 1984; Janvier, 1996b).
The hypophysial duct is placed slightly posteriorly in respect of the pineal gland and project anteroventrally, opening through the palate with a single foramen in the parasphenoid. This contrasts with what can be observed in arthrodire and crown-gnathostomes, where the hypophysial duct extends posteroventrally (Janvier, 1996b). Instead, it is common in jawless fish and some placoderms, as Brindabellaspis and Romundina, as well as petalichthyids, maybe signalling a separation between these forms and the other placoderms.

\paragraph{Midbrain}

The midbrain is the central part of the brain comprising the mesencephalon, located between the diencephalon and the subsequent metencephalon. Its anterior limit is not clear, but could be placed behind the optic nerve and the hypophysial duct. Its posterior edge, where it meets the cerebellar auricles of the metencephalon, is marked by an evident constriction. In the examined specimen the dorsal part of the brain cavity is broken, so that is not possible to have a precise picture of that area. Anyhow, it seem that distinct optic lobes where not present. Interestingly, the midbrain carries only the oculomotor nerve (III), while the trochlear nerve is instead associated with the cerebellum, a similar condition of that of Macropetalichthyis, lampreys and osteostracans (Stensiӧ, 1963, 1969; Nieuwenhuys, 1977; Janvier, 2008). This contrasts with the morphology of galeaspids, arthrodires, Romundina and most crown gnathostomes, where the trochlear nerve branches out from the midbrain near to the oculomotor nerve (Goujet, 1984; Elliott and Carr, 2010; Pradel, 2010; Gai et al., 2011; Dupret et al., 2014; Giles and Friedman, 2014) and could be phylogenetically significant (I still need a deep thinking about that.)

\paragraph{Hindbrain}

The hindbrain is well preserved so that the bounds of his two divisions, the metencephalon and the myelencephalon, could be easily detected. In its overall shape it resembles that of Brindabellaspis, with a metencephalon constituted by two symmetrical rounded lobes, the auricular recesses of the cerebellum, and a straight myelencephalic portion. The limit between these two divisions is marked by the posterior edge of the metencephalic lobes, where the brain become narrower and the facial nerve (VII) originated. From the metencephalon originate the trigeminal (V) and abducens nerves (VI) as in the other vertebrates, and, as exceptional for lampreys, osteostracans and petalichthyids, the trochlear nerve. The myelencephalon is the point of origin of the remaining cranial nerves and runs posteriorly until the end of the brain cavity. Unfortunately, in this specimen is only preserved the incomplete left  anterior part.
As note by (Janvier, 1985), the metencephalon of “primitive placoderms” as Brindabellaspis resemble that of jawless fishes in being located between the anterior part of the labyrinth cavity. This condition is present also in Shearsbyaspis, Macropetalichthys (Stensiӧ, 1963, 1969) and Romundina (Dupret et al., 2014), where the trigeminal recess branches out from the metencephalon at the same level as or near to the anterior ampulla. By contrast, in arthrodires (Stensiӧ, 1963, 1969; Young, 1979; Goujet, 1984; Dupret, 2010; Elliott and Carr, 2010), the trigeminal recess is located far anteriorly from the labyrinth. The first state has been considered a primitive feature of gnathostomes.

\paragraph{Labyrinth cavity}

Parts of both the labyrinth cavities are preserved in P33580, even if none of them is complete. The right labyrinth shows the position and morphology of the vestibule apparatus and related cranial nerves entering in it. As in Macropetalichthys, the saccular portion is very big compare to other placoderms, being as big as the orbits, if not more. A comparable proportion between the dimension of the labyrinth and the orbits could be seen in osteostracans and galeaspids (Janvier, 1985; Gai et al., 2011). Furthermore, the orbit and the labyrinth are very nearly placed, almost in contact each other, as in osteostracans, as the facial nerve passes so close to the labyrinth, before entering in the ventral posterior orbit, that there is almost indistinguable from the saccular wall. Another features shared with this jawless forms is a deep sacculus, also evident in the acanthothoracid Brindabellaspis and Romundina, in contrast with the ovoidal and narrower sacculus of arthrodire (Stensiӧ, 1969; Young, 1980; Goujet, 1984; Janvier, 1996b; Dupret et al., 2014).
The anterior ampulla is placed oddly ahead, behind the posterior external side of the orbit, so that the anterior semicircular circular is very long. The latter crosses the sacculus running towards the centre and enters in it above the point of connection between the labyrinth and the acoustic nerve (VIII). The anterior ampulla is connected to the external ampulla by a ventrally bended utriculus. Unfortunately the posteriormost part of the labyrinth is not preserved so it is not possible to observe the external semicircular canal nor the posterior ampulla and canal, as well the endolymphatic duct.

\subsubsection{Cranial nerves}
\paragraph{Terminal, olfactory and optic nerve}
Due to the general lengthening of telencephalon, the olfactory nerves are very long, more than three times the length of the telencephalon. They run anteroventrally from the olfactory lobes of the telencephalon to the nasal capsules, although the anterior part of the skull is missing and the position of these capsules cannot be observed. The olfactory nerve canals are also quite wide for all their length, being slightly thinner in their anterior part. They are similar in morphology with the olfactory tracts of elasmobranch but this could be regarded as an homoplasic character. Instead, petalichthyids are different to other placoderms, which have usually short olfactory tracts, not well-developed olfactory lobes and olfactory bulbs placed near the telencephalon (Janvier, 1996b).

Another single canal projects from the ventral portion of the telencephalon, running anteroventrally between the olfactory nerves. It could probably be associated with the terminal nerve, which occurs in all gnathostomes and has been observed in other placoderms as \textit{Brindabellaspis} and \textit{Romundina}, and in the galeaspids \textit{Shuyu}. In these forms, however, the terminal nerve has been described as a pair of canals, each of which exit from the olfactory lobes. In\textit{Shearsbyaspis}, instead, the terminal nerve runs as a single canal for almost all its length, bifurcating only at his anterior end.

Several small canals branch out from the olfactory and the terminal nerve. These has been observed also for Macropetalichthys and related by Stensiӧ (1925, 1963, 1969) to blood and nutritive canals. However, is evident form our CT scan that these canals connect the nerve with the dermal bone of the skull and thus it is possible that they have a sensory function.
The optic fenestra is very large, opening in the centre of the mesial orbital wall, anterodorsally to the eye-stalk. As in the other stem-gnathostomes, the optic nerve is the most anterior nerve connected with the eyes, as there is no evidence of muscle attachments anterior to it, as instead present in jawed vertebrates.

\paragraph{Oculomotor, trochlear nerves and abduscens}

The oculomotor nerve could be reconstructed in both sides. It arises from the posterior part of the mesencephalon, entering in the mesial wall of the orbit posteriorly to the optic nerve in a small foramen (3; 3a in Young 1985, fig. 6, p. 126) probably connected to the anterior oblique eye muscle. Two other canals branch out from this nerve, one opening posterodorsally to the optic foramen (3a), for the internal rectus, and the other (3p) running posteriorly on the external wall of the orbit and opening behind the eyestalk, possibily connected to the inferior rectus.

\section{Discussion}

\subsubsection{Dermal bone pattern}

The presence of pineal and rostral plates completely separating the preorbitals was originally assigned as a diagnostic feature for this Shearsbyaspis \citep{Young1985}. However, this condition has been observed also in \textit{Holopetalichthys} \citep{Zhu1996a}, \textit{Eurycaraspis} \citep{Liu1991} and \textit{Pauropetalichthys} \citep{pan2015new}. Furthermore, a separation of the two preorbital plates by rostropineal elements is common in petalichthyids, as present also in \textit{Macropetalichthys} \citep{Stensi1925}, \textit{Wijdeaspis} \citep{Young1978}, \textit{Quasipetalichthys} \citep{Liu1973} and \textit{Diandongpetalichthys} \citep{Zhu1991}, and also in several non petalichthyid placoderms, as the “acanthothoracid” \textit{Brindabellaspis} \citep{Young1980}, \textit{Romundina} \citep{dupret2011skull}, \textit{“Radotina prima”} \citep{dupret2011skull}, and the arthrodires \textit{Antarctaspis} \citep{white1968devonian} and \textit{Yujiangolepis} \citep{Dupret2009}. It might thus represents a primitive or homoplasic character.

The morphology of the anterior end of the skull-roof remains uncertain. Young (1985, figure 5), reconstructed the rostral plate as projecting out from the anterior margin of the skull, as in \textit{Notopetalichthys} \citep{Woodward1941} and \textit{Wijdeaspis} \citep{Young1978}. However, our CT scan reveals that the nasal capsules are not preserved and thus the skull would have extended further forward from what is preserved now. Furthermore, the position of the olfactory nerve seems to indicate that the nasal capsules were positioned laterally with respect to the rostral plates. A projecting rostral plate is also present in \textit{Lunaspis} \citep{gross1961lunaspis}, but is flanked by postnasal plates. It is interesting to note that in all these four taxa there is a shallow embayment where the supraorbital canal reaches the anterior margin of the preorbital plate. It is plausible to hypothesize that similar postnasal plates also flanked the rostral in \textit{Shearsbyaspis} \citep{Young1985}, and possibly in \textit{Notopetalichthys} \citep{Woodward1941} and \textit{Wijdeaspis} \citep{Young1978}, and have been taphonomically lost.

\section{Conclusion}

\section{Acknowledgements}


\bibliographystyle{apalike}
\bibliography{References}

\end{document}